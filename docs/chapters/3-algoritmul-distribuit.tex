\section{Algoritmul Distribuit: Protocolul Kerberos}

Kerberos rezolvă problema autentificării într-o rețea nesigură folosind un \textit{Trusted Third Party} (KDC). Pentru a face protocolul ușor de înțeles, îl putem privi ca pe un sistem de bilete: utilizatorul nu prezintă parola la fiecare serviciu, ci obține un "permis universal" (TGT) pe care îl schimbă ulterior pentru "bilete specifice" de acces.

\subsection{Notații Utilizate}
Pentru descrierea formală, vom folosi următoarele simboluri simplificate, conform lucrării originale \cite{steiner1988kerberos}:

% Nu mai închide multicols aici

\begin{table*}[t] % table* forțează afișarea pe toată lățimea, [t] îl pune sus
    \centering
    \begin{tabular}{|c|l|}
        \hline
        \textbf{Simbol} & \textbf{Semnificație}                                             \\
        \hline
        $C, S$          & Clientul (utilizator) și Serverul (serviciul dorit)               \\
        $AS, TGS$       & Componentele KDC: Authentication Server și Ticket-Granting Server \\
        $K_x$           & Cheia secretă a lui $x$ (ex: hash-ul parolei)                     \\
        $\{M\}_K$       & Mesajul $M$ criptat cu cheia $K$                                  \\
        $TGT$           & \textit{Ticket-Granting Ticket} (Permisul universal)              \\
        $T_{Serv}$      & Tichetul de Serviciu (pentru acces final)                         \\
        $Auth$          & Autentificatorul (dovada identității pe loc, conține timestamp)   \\
        \hline
    \end{tabular}
    \caption{Notații și Simboluri Kerberos}
    \label{tab:notatii}
\end{table*}

% Continuă textul normal în două coloane...

\subsection{Fluxul Protocolului (Cele 3 Faze)}

Protocolul este împărțit în trei etape logice.

\subsubsection{Faza 1: Autentificarea Inițială (Logarea)}
\textbf{Scop:} Clientul vrea să demonstreze cine este, fără a trimite parola prin rețea.

În această fază, clientul inițiază protocolul pentru a obține credențialele necesare interacțiunii ulterioare. Figura de mai jos ilustrează schimbul de mesaje, unde axa timpului curge de sus în jos.

\begin{figure}[H]
    \centering
    % \resizebox forțează diagrama să se micșoreze la lățimea coloanei curente
    \resizebox{\linewidth}{!}{
        \centering
        \begin{tikzpicture}[
        node distance=3.2cm, % Distanță compactă, identică cu TGS
        % --- DEFINIȚII DE STIL (Copy-Paste din TGS pentru consistență) ---
        % 1. Actorii (Client/Server): Gradient Albastru + Umbră
        entity/.style={
                draw=blue!40!black,
                top color=white,
                bottom color=blue!10,
                rounded corners=3pt,
                align=center,
                minimum height=1cm,
                minimum width=1.8cm,
                drop shadow,
                font=\bfseries\small
            },
        % 2. Mesajele de pe săgeți
        msg/.style={
                midway,
                above,
                font=\scriptsize,
                color=black!80
            },
        % 3. Notele Interne (Chenar): Galben pal + Border Portocaliu
        process/.style={
                draw=orange!50!black,
                fill=orange!5,
                rounded corners=2pt,
                align=left,
                font=\tiny\sffamily,
                inner sep=4pt,
                drop shadow={opacity=0.3}
            },
        % 4. Formulele Matematice de sub săgeți
        mathnote/.style={
                below=0.1cm,
                font=\tiny,
                color=black!70
            }
    ]

    % 1. Actorii
    \node[entity] (client) {Client ($C$)};
    \node[entity, right=of client] (as) {Auth Server};

    % 2. Liniile vieții
    \draw[thick, gray!50] (client) -- ++(0,-5.5) coordinate (c_end);
    \draw[thick, gray!50] (as) -- ++(0,-5.5) coordinate (as_end);

    % 3. Mesajul 1 (Cerere)
    \draw[-{Stealth[length=3mm]}, thick, color=blue!40!black]
    ($(client)+(0,-1)$) --
    node[msg] {1. Cerere: $ID_C, ID_{TGS}$}
    ($(as)+(0,-1.1)$);

    % --- CHENARUL CERUT (Procesare Internă AS) ---
    % Folosim stilul 'process'
    \node[process, below=0.3cm of as, anchor=north] {
        $\bullet$ Verificare în DB\\
        $\bullet$ Generare $K_{Sess}$\\
        $\bullet$ Creare $TGT$
    };

    % 4. Mesajul 2 (Răspuns)
    \draw[-{Stealth[length=3mm]}, thick, color=blue!40!black]
    ($(as)+(0,-4)$) --
    node[msg, align=center] {2. Răspuns: $\{K_{Sess}\}_{K_C}$ \\ $+ TGT$}
    ($(client)+(0,-4)$);

    % Explicatie Decriptare (Stânga)
    \node[mathnote, align=center] at ($(client)+(0,-4)$) {
        \textit{*Decriptează}\\
        \textit{pt. $K_{Sess}$}
    };

    % Explicatie TGT (Dreapta) - Poziționată pe mijlocul distanței pentru simetrie
    \node[mathnote] at ($(client)!0.5!(as) + (0,-4)$) {
    $TGT = \{C, K_{Sess}, Life\}_{K_{TGS}}$
    };

\end{tikzpicture}
    } % <--- Aceasta este acolada care lipsea!
    \caption{Diagrama de secvență pentru Autentificarea Inițială (AS Exchange)}
    \label{fig:as_exchange}
\end{figure}

După acest schimb, clientul deține cheia de sesiune decriptată și TGT-ul opac, fiind pregătit pentru Faza 2.


\subsubsection{Faza 2: Obținerea Tichetului de Serviciu (TGS Exchange)}
\textbf{Scop:} Clientul are TGT-ul (permisul) și vrea să acceseze un server specific ($S$), de exemplu un Server de Fișiere.

Utilizând cheia de sesiune obținută anterior, clientul generează un \textit{Autentificator} pentru a dovedi că este posesorul legitim al TGT-ului.

\begin{figure}[H]
    \centering
    % Resizebox asigură că diagrama încape perfect în coloană
    \resizebox{\linewidth}{!}{
        \begin{tikzpicture}[
        node distance=3.2cm, % Distanță compactă
        % --- DEFINIȚII DE STIL (Aici e magia design-ului) ---
        % 1. Actorii (Client/Server): Gradient Albastru + Umbră
        entity/.style={
                draw=blue!40!black,
                top color=white,
                bottom color=blue!10,
                rounded corners=3pt,
                align=center,
                minimum height=1cm,
                minimum width=1.8cm,
                drop shadow,
                font=\bfseries\small
            },
        % 2. Mesajele de pe săgeți
        msg/.style={
                midway,
                above,
                font=\scriptsize,
                color=black!80
            },
        % 3. Notele Interne (Chenarul cerut): Galben pal + Border Portocaliu
        process/.style={
                draw=orange!50!black,
                fill=orange!5,
                rounded corners=2pt,
                align=left,
                font=\tiny\sffamily, % Font sans-serif pentru cod/tehnic
                inner sep=4pt,
                drop shadow={opacity=0.3}
            },
        % 4. Formulele Matematice de sub săgeți
        mathnote/.style={
                below=0.1cm,
                font=\tiny,
                color=black!70
            }
    ]

    % 1. Actorii
    \node[entity] (client) {Client ($C$)};
    \node[entity, right=of client] (tgs) {TGS Server};

    % 2. Liniile vieții
    \draw[thick, gray!50] (client) -- ++(0,-5.5) coordinate (c_end);
    \draw[thick, gray!50] (tgs) -- ++(0,-5.5) coordinate (tgs_end);

    % 3. Mesajul 3 (Cerere)
    \draw[-{Stealth[length=3mm]}, thick, color=blue!40!black]
    ($(client)+(0,-1.1)$) --
    node[msg] {3. Cerere: $S, TGT, Auth_C$}
    ($(tgs)+(0,-1.1)$);

    % Formula Auth
    \node[mathnote] at ($(client)!0.5!(tgs) + (0,-1)$) {
    $Auth_C = \{C, TS\}_{K_{Sess}}$
    };

    % --- CHENARUL CERUT (Procesare Internă) ---
    % Folosim stilul 'process' definit mai sus
    \node[process, below=0.5cm of tgs, anchor=north] {
        $\bullet$ Decr. TGT ($K_{TGS}$)\\
        $\bullet$ Verif. Auth ($K_{Sess}$)\\
        $\bullet$ Gen. $K_{C,S}$ \& $T_{Serv}$
    };

    % 4. Mesajul 4 (Răspuns)
    \draw[-{Stealth[length=3mm]}, thick, color=blue!40!black]
    ($(tgs)+(0,-4)$) --
    node[msg, align=center] {4. Răspuns: $\{K_{C,S}, S\}_{K_{Sess}}$ \\ $+ T_{Serv}$}
    ($(client)+(0,-4)$);

    % Formula T_Serv
    \node[mathnote] at ($(client)!0.5!(tgs) + (0,-4)$) {
    $T_{Serv} = \{C, K_{C,S}, Life\}_{K_S}$
    };

\end{tikzpicture}

    }
    \caption{Diagrama de secvență pentru TGS Exchange}
    \label{fig:tgs_exchange}
\end{figure}

TGS-ul verifică timestamp-ul din autentificator pentru a preveni atacurile de tip \textit{replay}. Dacă totul este valid, clientul primește noul tichet ($T_{Serv}$) și noua cheie de sesiune ($K_{C,S}$).

\subsubsection{Faza 3: Accesarea Serviciului (Client-Server Exchange)}
\textbf{Scop:} Clientul prezintă tichetul final serverului.

\begin{figure}[H]
    \centering
    % \resizebox asigură scalarea automată la lățimea coloanei
    \resizebox{\linewidth}{!}{
        \begin{tikzpicture}[
        node distance=3.2cm, % Același spațiere compactă
        % --- STILURI (Identice cu celelalte figuri) ---
        entity/.style={
                draw=blue!40!black,
                top color=white,
                bottom color=blue!10,
                rounded corners=3pt,
                align=center,
                minimum height=1cm,
                minimum width=1.8cm,
                drop shadow,
                font=\bfseries\small
            },
        msg/.style={
                midway,
                above,
                font=\scriptsize,
                color=black!80
            },
        process/.style={
                draw=orange!50!black,
                fill=orange!5,
                rounded corners=2pt,
                align=left,
                font=\tiny\sffamily,
                inner sep=4pt,
                drop shadow={opacity=0.3}
            },
        mathnote/.style={
                below=0.1cm,
                font=\tiny,
                color=black!70
            }
    ]

    % 1. Actorii
    \node[entity] (client) {Client ($C$)};
    \node[entity, right=of client] (server) {Server ($S$)};

    % 2. Liniile vieții
    \draw[thick, gray!50] (client) -- ++(0,-5) coordinate (c_end);
    \draw[thick, gray!50] (server) -- ++(0,-5) coordinate (s_end);

    % 3. Mesajul 5 (Cerere Acces)
    \draw[-{Stealth[length=3mm]}, thick, color=blue!40!black]
    ($(client)+(0,-1.1)$) --
    node[msg] {5. Cerere: $T_{Serv}, Auth_C$}
    ($(server)+(0,-1.1)$);

    % Formula Auth
    \node[mathnote] at ($(client)!0.5!(server) + (0,-1)$) {
    $Auth_C = \{C, TS\}_{K_{C,S}}$
    };

    % --- CHENAR PROCESARE INTERNĂ SERVER ---
    % Am adăugat '\\' la final de rând pentru a forța verticalitatea
    \node[process, below=0.5cm of server, anchor=north] {
        $\bullet$ Decr. $T_{Serv}$ ($K_S$)\\
        $\bullet$ Extrage $K_{C,S}$\\
        $\bullet$ Verif. $TS$ (< 5min)
    };

    % 4. Mesajul 6 (Confirmare / Acces)
    \draw[-{Stealth[length=3mm]}, thick, color=blue!40!black]
    ($(server)+(0,-4)$) --
    node[msg, align=center] {6. Răspuns: Acces Permis \\ (sau $\{TS+1\}_{K_{C,S}}$)}
    ($(client)+(0,-4)$);

    % Notă finală
    \node[mathnote] at ($(client)!0.5!(server) + (0,-4)$) {
        \textit{*Sesiune stabilită}
    };

\end{tikzpicture}
    }
    \caption{Diagrama Client-Server Exchange}
    \label{fig:cs_exchange}
\end{figure}

\subsection{De ce este acest sistem sigur?}
\begin{enumerate}
    \item \textbf{Parola nu circulă niciodată:} Ea este folosită doar local pe stația clientului pentru a decripta primul mesaj de la AS.
    \item \textbf{Protecție la Replay:} Datorită Autentificatorului care conține \textit{Timestamp}, un atacator care interceptează pachetele nu le poate refolosi mai târziu, deoarece serverul va respinge cererile cu ora veche.
    \item \textbf{Minimizarea riscului:} Serverul de aplicație nu trebuie să știe parola utilizatorului, ci doar să poată decripta tichetul emis de KDC.
\end{enumerate}

\subsection{Replicarea și Administrarea Datelor}
Pentru a asigura disponibilitatea și toleranța la defecte (fault tolerance), Kerberos nu rulează pe un singur nod, ci folosește o arhitectură distribuită de tip \textbf{Master-Slave} pentru baza de date de credențiale.

\subsubsection{Arhitectura Master-Slave}
Baza de date este replicată pe mai multe servere KDC:
\begin{itemize}
    \item \textbf{Master KDC:} Este singurul nod care permite operații de scriere (adăugare utilizatori, schimbare parole). Acesta gestionează copia autoritativă a bazei de date.
    \item \textbf{Slave KDCs:} Sunt noduri \textit{Read-Only}. Acestea primesc periodic actualizări de la Master. Clientul se poate autentifica la orice Slave disponibil. Dacă Master-ul pică, autentificarea funcționează în continuare prin Slave-uri, dar nu se pot schimba parole.
\end{itemize}

Această separare este crucială pentru performanța sistemului distribuit, deoarece cererile de citire (autentificările frecvente) sunt balansate între multiple replici, în timp ce scrierile (rare) sunt centralizate pentru consistență.

\subsubsection{Serviciul de Administrare (KDBM)}
Modificările asupra bazei de date se fac exclusiv prin protocolul administrativ, gestionat de procesul \texttt{kadmind}. Deoarece baza de date conține cheile secrete ale tuturor utilizatorilor, acest serviciu rulează doar pe nodul Master și necesită o autentificare strictă.
