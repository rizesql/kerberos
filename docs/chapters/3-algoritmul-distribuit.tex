\section{Algoritmul Distribuit: Protocolul Kerberos}

Kerberos este un protocol de autentificare bazat pe conceptul de \textit{Trusted Third Party}. Pentru a descrie interacțiunea distribuită dintre entități, vom utiliza notația standard consacrată de Steiner, Neuman și Schiller \cite{steiner1988kerberos}.

\subsection{Notații și Simboluri}

Pentru claritatea descrierii formale, definim următorii termeni și abrevieri:

\end{multicols}
\begin{center}
    \begin{tabular}{|c|l|}
        \hline
        \textbf{Simbol} & \textbf{Descriere}                                       \\
        \hline
        $c$             & Clientul (principalul care solicită acces)               \\
        $s$             & Serverul (serviciul accesat, ex: File Server)            \\
        $AS$            & Authentication Server (componentă KDC)                   \\
        $TGS$           & Ticket-Granting Server (componentă KDC)                  \\
        $K_x$           & Cheia secretă a utilizatorului $x$ (derivată din parolă) \\
        $K_{x,y}$       & Cheia de sesiune temporară între $x$ și $y$              \\
        $\{m\}_K$       & Mesajul $m$ criptat simetric cu cheia $K$                \\
        $T_{x,y}$       & Tichetul (Ticket) emis pentru $x$ să acceseze $y$        \\
        $A_x$           & Autentificatorul (Authenticator) generat de client       \\
        $TS$            & Timestamp (pentru prevenirea atacurilor \textit{replay}) \\
        \hline
    \end{tabular}
\end{center}
\begin{multicols}{2}

\subsection{Fluxul de Execuție (Model SPMD)}

Protocolul funcționează asincron, prin schimburi de mesaje inițiate de client. Acesta poate fi descompus în trei faze logice distincte.

% --- FAZA 1 ---
\subsubsection{Faza 1: Autentificarea Inițială (AS Exchange)}
În această etapă, clientul își dovedește identitatea față de KDC fără a transmite parola prin rețea.

\textbf{Pasul 1 (Cererea):} Clientul trimite un mesaj în clar către AS, solicitând acces la TGS.
\begin{equation}
    c \rightarrow AS: \quad ID_c, ID_{tgs}, TS_1
\end{equation}

\textbf{Pasul 2 (Răspunsul):} AS verifică identitatea în baza de date. Dacă este validă, generează o cheie de sesiune $K_{c,TGS}$ și răspunde cu două pachete de date:
\begin{equation}
    AS \rightarrow c: \quad \{K_{c,TGS}, TS_2, Life, TGS\}_{K_c}, \quad T_{c,TGS}
\end{equation}

\textit{Nota bene:} Clientul poate decripta primul pachet doar dacă introduce parola corectă (care generează $K_c$). Al doilea pachet, $T_{c,TGS}$ (Ticket-Granting Ticket), este criptat cu cheia secretă a TGS și este opac pentru client.

% --- FAZA 2 ---
\subsubsection{Faza 2: Obținerea Tichetului de Serviciu (TGS Exchange)}
Clientul deține acum un TGT și vrea să acceseze un serviciu specific $s$.

\textbf{Pasul 3 (Cererea):} Clientul prezintă TGT-ul și un Autentificator proaspăt generat.
\begin{equation}
    c \rightarrow TGS: \quad s, \quad T_{c,TGS}, \quad A_c
\end{equation}
Unde $A_c = \{c, TS_3\}_{K_{c,TGS}}$. Acest lucru dovedește că cel care trimite tichetul este posesorul legitim al cheii de sesiune.

\textbf{Pasul 4 (Răspunsul):} TGS decriptează TGT-ul, verifică validitatea și emite un tichet pentru serverul final.
\begin{equation}
    TGS \rightarrow c: \quad \{K_{c,s}, s, TS_4\}_{K_{c,TGS}}, \quad T_{c,s}
\end{equation}

% --- FAZA 3 ---
\subsubsection{Faza 3: Accesarea Serviciului (CS Exchange)}
Clientul are acum un tichet valid pentru serverul $s$.

\textbf{Pasul 5 (Autentificarea):} Clientul trimite tichetul și un nou autentificator către server.
\begin{equation}
    c \rightarrow s: \quad T_{c,s}, \quad A_c'
\end{equation}

\textbf{Pasul 6 (Confirmarea Mutuală - Opțional):} Serverul confirmă identitatea sa clientului, demonstrând că a putut decripta tichetul și a extras timestamp-ul.
\begin{equation}
    s \rightarrow c: \quad \{TS_5 + 1\}_{K_{c,s}}
\end{equation}

Această arhitectură distribuie încrederea: $AS$ garantează identitatea inițială, iar $TGS$ garantează dreptul de acces la resurse, minimizând expunerea cheilor pe termen lung.