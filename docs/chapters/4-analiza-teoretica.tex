\section{Analiză Teoretică}

Conform cerințelor de proiectare, evaluăm protocolul din perspectiva corectitudinii, performanței și scalabilității.

% CERINȚA 1: Analiză corectitudine
\subsection{Analiză Corectitudine}
Corectitudinea protocolului Kerberos se bazează pe două premise esențiale care diferențiază acest sistem distribuit de varianta serială:

\begin{enumerate}
    \item \textbf{Sincronizarea Ceasurilor (Time Skew):} Deoarece autentificatorii ($Auth_C$) se bazează pe timestamp-uri pentru a preveni atacurile de tip \textit{replay}, este critic ca ceasurile clientului și serverului să fie sincronizate (de obicei prin protocolul NTP). O deviație mai mare de 5 minute (configurație standard) duce la respingerea tichetelor valid emise, cauzând o eroare de disponibilitate (False Negative).
    \item \textbf{Integritatea TTP (Trusted Third Party):} Securitatea întregului sistem depinde de faptul că KDC-ul nu este compromis. Deoarece KDC-ul stochează cheile secrete ale tuturor principalilor, compromiterea acestuia permite unui atacator să genereze orice tichet (atac de tip "Golden Ticket"), permițând impersonarea oricărui utilizator către orice serviciu.
\end{enumerate}

% CERINȚA 2: Analiză complexitate timp / mesaj
\subsection{Analiză Complexitate Timp / Mesaj}
Pentru o sesiune completă de autentificare și acces la un serviciu ("Cold Start"), complexitatea mesajelor este constantă, totalizând 6 pachete de rețea:

\begin{itemize}
    \item \textbf{AS Exchange:} 2 mesaje (Cerere + Răspuns).
    \item \textbf{TGS Exchange:} 2 mesaje (Cerere + Răspuns).
    \item \textbf{Client-Server:} 2 mesaje (Cerere + Confirmare).
\end{itemize}

\textbf{Optimizare (Cache):} Ulterior, pentru accesarea aceluiași serviciu, costul scade la \textbf{2 mesaje}, deoarece clientul refolosește tichetul de serviciu ($T_{Serv}$) până la expirarea acestuia (uzual 8 ore). Pentru accesarea unui \textit{serviciu nou}, costul este de \textbf{4 mesaje} (TGS + CS), deoarece se refolosește TGT-ul.

Din punct de vedere al complexității comunicației, protocolul este extrem de eficient, având un număr constant de pași (\textbf{$O(1)$}), indiferent de numărul total de utilizatori din sistem.
Din punct de vedere computațional, încărcarea asupra procesorului este redusă, deoarece protocolul utilizează exclusiv criptare simetrică (DES/AES), care este semnificativ mai rapidă decât operațiile matematice complexe necesare în criptarea asimetrică (RSA/ECC).

% CERINȚA 3: Avantaje / dezavantaje cu variația numărului de noduri
\subsection{Comportamentul la Variația Numărului de Noduri (Replicarea Master-Slave)}

Scalarea sistemului prin adăugarea de noduri suplimentare (Slave KDCs) prezintă următoarele caracteristici tehnice:

\begin{itemize}
    \item \textbf{Avantaje (Read Throughput):}
          Adăugarea de servere Slave crește liniar capacitatea de procesare a cererilor de autentificare (AS/TGS). Deoarece majoritatea traficului în Kerberos este de tip \textit{Read-Only} (verificarea credențialelor și emiterea de tichete), sistemul scalează eficient pentru un număr mare de utilizatori simultani.

    \item \textbf{Dezavantaje (Write Bottleneck \& Propagation Lag):}
          Indiferent de numărul de noduri Slave adăugate, arhitectura impune o limitare critică:
          \begin{enumerate}
              \item \textbf{Scrieri Exclusive pe Master:} Orice modificare a bazei de date (schimbarea parolei, adăugarea unui utilizator) trebuie procesată \textbf{exclusiv} de KDC-ul Master. Astfel, Master-ul rămâne un \textit{Single Point of Failure} (SPOF) pentru operațiunile de scriere. Dacă Master-ul cade, utilizatorii se pot autentifica (prin Slave), dar nu își pot schimba parolele.
              \item \textbf{Latența de Propagare:} Există un decalaj (\textit{propagation lag}) între momentul scrierii pe Master și actualizarea tuturor Slave-urilor. În acest interval, un utilizator ar putea să se autentifice pe un Slave folosind vechea parolă, deși a schimbat-o pe Master.
          \end{enumerate}
\end{itemize}
% CERINȚA 4: Exemplificare topologii
\subsection{Exemplificare Topologii}
Pe lângă arhitectura standard (Client-Server-KDC), protocolul suportă topologii complexe pentru medii enterprise.

\subsubsection{Topologie Cross-Realm}
Aceasta permite unui client dintr-un domeniu (Realm A) să acceseze servicii dintr-un alt domeniu (Realm B), cu condiția să existe o relație de încredere (cheie partajată) între cele două servere KDC.

\begin{figure*}[t]
    \centering
    \resizebox{0.9\textwidth}{!}{
        \begin{tikzpicture}[
    node distance=2.5cm,
    % --- STILURI VIZUALE ---
    kdc/.style={
            draw=blue!40!black,
            top color=white,
            bottom color=blue!15,
            rounded corners=3pt,
            align=center,
            minimum height=1.5cm,
            minimum width=2.5cm,
            drop shadow,
            font=\bfseries
        },
    entity/.style={
            draw=gray!60!black,
            top color=white,
            bottom color=gray!10,
            rectangle,
            rounded corners,
            minimum height=1cm,
            minimum width=2cm,
            align=center,
            drop shadow,
            font=\small
        },
    flow/.style={
    ->,
    >={Stealth[length=3mm]},
    thick,
    color=blue!40!black,
    rounded corners
    },
    lbl/.style={
            midway,
            fill=white,
            inner sep=1.5pt,
            font=\scriptsize\bfseries,
            text=blue!40!black,
            align=center,
            draw=blue!10,
            rounded corners=2pt
        },
    realm/.style={
            draw,
            dashed,
            rounded corners=10pt,
            inner sep=15pt,
            thick
        }
    ]

    % --- POZIȚIONARE NODURI ---
    % REALM A (Stânga)
    \node[kdc] (kdcA) at (0, 4) {KDC A\\(Local Realm)};
    \node[entity] (client) at (0, 0) {Client @ A};

    % REALM B (Dreapta)
    \node[kdc] (kdcB) at (10, 4) {KDC B\\(Remote Realm)};
    \node[entity] (server) at (10, 0) {Server @ B};

    % --- FUNDALURI REALMURI ---
    \begin{scope}[on background layer]
        \node[realm, draw=red!40, fill=red!5, fit=(kdcA) (client)] (boxA) {};
        \node[anchor=north west, color=red!60!black, font=\bfseries] at (boxA.north west) {Realm A};

        \node[realm, draw=green!40!black, fill=green!5, fit=(kdcB) (server)] (boxB) {};
        \node[anchor=north east, color=green!40!black, font=\bfseries] at (boxB.north east) {Realm B};
    \end{scope}

    % --- CONEXIUNI ȘI FLUX CORECTAT ---

    % 0. TRUST LINK
    \draw[<->, dashed, ultra thick, color=orange!60!black] (kdcA) --
    node[midway, above, font=\bfseries\small, color=orange!60!black] {Inter-Realm Trust Key}
    (kdcB);

    % 1. Cerere TGT pentru Remote
    \draw[flow] (client) --
    node[lbl, left] {1. Cere TGT\\pentru Realm B}
    (kdcA);

    % 2. Prezentare TGT la KDC B
    \draw[flow] (client) --
    node[lbl, sloped, above] {2. Prezintă Remote TGT\\(Obține $T_{Serv}$)}
    (kdcB);

    % 3. Acces Final (CORECTAT: Client -> Server, nu KDC -> Server)
    % Tichetul este prezentat de Client
    \draw[flow] (client) --
    node[lbl, below] {3. Acces Final\\cu $T_{Serv}$}
    (server);

\end{tikzpicture}

    }
    \caption{Topologie Cross-Realm: Autentificare între domenii diferite}
    \label{fig:cross_realm}
\end{figure*}

În figura \ref{fig:cross_realm}, fluxul este modificat astfel:
\begin{enumerate}
    \item Clientul cere KDC-ului local un tichet pentru KDC-ul străin.
    \item KDC-ul local emite un "Remote TGT" criptat cu cheia inter-realm.
    \item Clientul prezintă acest TGT direct KDC-ului străin și primește tichetul de serviciu ($T_{Serv}$).
    \item Clientul prezintă $T_{Serv}$ serverului final pentru acces.
\end{enumerate}

\subsubsection{Topologie de Delegare (Proxy)}
O inovație majoră descrisă în lucrarea originală Athena \cite{steiner1988kerberos} este mecanismul de \textit{Proxy}. Aceasta rezolvă problema în care un serviciu intermediar trebuie să acceseze o resursă în numele utilizatorului.

\begin{figure}[H]
    \centering
    \resizebox{\linewidth}{!}{
        \begin{tikzpicture}[
    node distance=2.5cm,
    % --- STILURI (Compacte pentru o coloană) ---
    entity/.style={
            draw=blue!40!black,
            top color=white,
            bottom color=blue!10,
            rounded corners=3pt,
            align=center,
            minimum height=1cm,
            minimum width=2cm,
            drop shadow,
            font=\bfseries\small
        },
    flow/.style={
    ->,
    >={Stealth[length=3mm]},
    thick,
    color=blue!40!black
    },
    lbl/.style={
            midway,
            fill=white,
            inner sep=1pt,
            font=\tiny\sffamily,
            text=blue!50!black,
            align=center,
            draw=blue!10,
            rounded corners=2pt
        }
    ]

    % --- POZIȚIONARE PE VERTICALĂ (Flux Liniar) ---
    \node[entity] (client) {Client\\(User)};
    \node[entity, below=1.5cm of client] (print) {Service A\\(ex: Print Server)};
    \node[entity, below=1.5cm of print] (file) {Service B\\(ex: File Server)};

    % --- FLUX ---

    % 1. Client -> Print
    \draw[flow] (client) -- node[lbl] {1. Trimite Tichet Proxy\\(Permisiune de citire)} (print);

    % 2. Print -> File
    % Mică rafinare la text pentru precizie maximă
    \draw[flow] (print) -- node[lbl] {2. Prezintă Tichetul\\(Cere acces fișier)} (file);

    % 3. Confirmare
    \draw[flow, dashed, color=gray] (file) to[out=160, in=200] node[lbl, left] {3. Transfer Date} (print);

\end{tikzpicture}
    }
    \caption{Topologie Proxy: Delegarea drepturilor între servicii}
    \label{fig:proxy_topology}
\end{figure}

Această arhitectură simplifică managementul accesului în sisteme distribuite complexe, eliminând necesitatea autentificării multiple.