\section{Algoritmul Serial}

În contextul autentificării, varianta serială a problemei presupune existența unui sistem monolitic în care utilizatorul și resursa accesată se află pe aceeași mașină fizică, sub controlul aceluiași nucleu (kernel). În acest scenariu, nu există o rețea de comunicații interpusă, iar mediul este considerat de încredere (Trusted Computing Base).

\subsection{Descrierea Problemei}
Problema fundamentală este verificarea identității unui \textit{principal} (utilizator sau serviciu) înainte de a-i acorda acces la resursele sistemului. Autentificarea se bazează pe ceva ce utilizatorul \textit{știe} (o parolă), care este comparată cu o valoare de referință stocată într-o bază de date securizată a sistemului.

\subsection{Descrierea Algoritmului}
Algoritmul serial de autentificare urmează un flux liniar de verificare a hash-urilor. Deoarece magistrala de date este internă și sigură, parola poate fi procesată direct, fără necesitatea unor tichete sau mecanisme de prevenire a atacurilor de tip \textit{replay}, esențiale în varianta distribuită.

Pașii algoritmului sunt:
\begin{enumerate}
    \item Utilizatorul furnizează identificatorul unic (\textit{UserID}) și parola (\textit{Password}).
    \item Sistemul caută în tabela de \textit{principals} (echivalentul \texttt{/etc/shadow} în UNIX) intrarea corespunzătoare identificatorului.
    \item Dacă identificatorul nu există, accesul este refuzat imediat pentru a preveni scurgerea de informații despre existența conturilor.
    \item Dacă utilizatorul există, se extrage cheia (sau hash-ul) stocată în baza de date (\texttt{key\_bytes}).
    \item Sistemul aplică funcția de derivare a cheii (KDF) asupra parolei introduse.
    \item Se compară bit cu bit rezultatul obținut cu valoarea stocată.
    \item Dacă valorile coincid, se inițializează o sesiune locală; altfel, se returnează o eroare de autentificare.
\end{enumerate}

\subsection{Pseudocod}
Algoritmul poate fi formalizat după cum urmează:

% Nu mai închide multicols (\end{multicols})

\begin{algorithm*}[t] % Folosește algorithm* cu [t] (top)
    \caption{Autentificare Serială (Locală)}
    \begin{algorithmic}[1]
        \Procedure{SerialAuth}{UserID, InputPassword}
        \State $record \gets \text{DB.lookup}(\text{primary\_name} = UserID)$
        \If{$record = \text{NULL}$}
        \Return \textbf{false} \Comment{Utilizator inexistent}
        \EndIf
        \State $DerivedKey \gets \text{KDF}(InputPassword)$
        \If{$DerivedKey = record.\text{key\_bytes}$}
        \State \text{CreateLocalSession}(UserID)
        \Return \textbf{true}
        \Else
        \Return \textbf{false} \Comment{Parolă incorectă}
        \EndIf
        \EndProcedure
    \end{algorithmic}
\end{algorithm*}

% Continuă textul normal, fără a redeschide multicols